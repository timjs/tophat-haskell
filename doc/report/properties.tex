% !TEX root=../report.tex

\section{Overview}

\statefultrue

\subsection{Equality}

\begin{align*}
\forall t_1, t_2 .  t_1 \sim t_2 \iff \forall i . &        & t_1 \handle{i} t_1'\\
                                                  & \wedge & t_2 \handle{i} t_2'\\
                                                  & \wedge & \Value{(t_1')} \equiv \Value{(t_2')}\\
                                                  & \wedge & t_1' \sim t_2'
\end{align*}

\subsection{Equivalence}

\begin{align*}
\forall t_1, t_2 .  t_1 \simeq t_2 \iff \forall i_1 .\exists i_2 &        & t_1 \handle{i_1} t_1'\\
                                                    & \wedge & t_2 \handle{i_2} t_2'\\
                                                    & \wedge & \Value{(t_1')} \equiv \Value{(t_2')}\\
                                                    & \wedge & t_1' \simeq t_2'\\
                                                    & \wedge & i_1 \bumpeq i_2\\
                                      \wedge\forall i_2.\exists i_1 & & t_2 \handle{i_2} t_2'\\
                                      & \wedge & t_1 \handle{i_1} t_1'\\
                                      & \wedge & \Value{(t_1')} \equiv \Value{(t_2')}\\
                                      & \wedge & t_1' \simeq t_2'\\
                                      & \wedge & i_1 \bumpeq i_2\\
\end{align*}

\subsection{Progress}

\begin{align*}
  \forall e:\tau,i & e \handle{i} e' &\implies& \text{ either Fail}(e') \text{ or } \exists e'',i' \text{ such that } e'\handle{i'}e''&\\
\end{align*}

\subsection{Preservation}

\begin{theorem}[preservation under handling]
  $\forall e:\tau,i  e \handle{i} e' \implies e':\tau$
\end{theorem}

\begin{theorem}[preservation under normalization]
    $\forall e:\tau  e \normalise e' \implies e':\tau$
\end{theorem}

\begin{theorem}[preservation under evaluation]
      $\forall e:\tau  e \evaluate e' \implies e':\tau$
      \label{thmpreseval}
\end{theorem}

\begin{lemma}
  $\forall e:\Task\tau . \Value{e}=v \implies v:\tau$
\end{lemma}

\begin{proof}
  We prove Theorem~\ref{thmpreseval} by induction on $e$:\\
  \noindent\textbf{Case} $e=\lambda x:\tau.e, e_1 e_2, x, c, l, e_1 \star e_2,
                            \If{e_1}{e_2}{e_3},\tuple{e_1, e_2},\unit,\Ref e,!e,
                            e_1 := e_2,e_1; e_2$
      preservation has been proven for these cases by \todo{insert cite}\\

  \noindent\textbf{Case} $\userule{E-Edit}$
      By T-Edit we have $\Edit e:\Task \tau$ and $e:\tau$. The induction
      hypothesis gives us that $e\evaluate v$ also preserves, and thus $v:\tau$.
      Therefore $\Edit v:\Task\tau$.\\

  \noindent\textbf{Case} $\userule{E-Fill}$
      Evaluation does not alter the expression, therefore this case holds tivially.\\

  \noindent\textbf{Case} $\userule{E-Update}$
      By T-Update we have $\Edit e:\Task \tau$ and $e:\Ref \tau$. The induction
      hypothesis gives us that $e\evaluate l$ also preserves, and thus $l:\Ref\tau$.
      Therefore $\Update l:\Task\tau$\\

  \noindent\textbf{Case} $\userule{E-Fail}$
      Evaluation does not alter the expression, therefore this case holds tivially.\\

  \noindent\textbf{Case} $\userule{E-Then}$
      Given that $e_1\Then e_2:\Task \tau$, T-Then gives us that $e_1:\Task\tau_1$
      and $e_2:\tau_1 \to \Task \tau$. By the induction hypothesis, we know that
      $e_1\evaluate t_1$ preserves and thus $t_1:\Task\tau_1$. Therefore
      $t_1\Then e_2:\Task\tau$.\\

  \noindent\textbf{Case} $\userule{E-Next}$
      Given that $e_1\Next e_2:\Task \tau$, T-Then gives us that $e_1:\Task\tau_1$
      and $e_2:\tau_1 \to \Task \tau$. By the induction hypothesis, we know that
      $e_1\evaluate t_1$ preserves and thus $t_1:\Task\tau_1$. Therefore
      $t_1\Next e_2:\Task\tau$.\\

  \noindent\textbf{Case} $\userule{E-And}$
      Given that $e_1\And e_2:\Task(\tau_1\times\tau_2)$, T-And gives us that
      $e_1:\Task\tau_1$ and $e_2:\Task\tau_2$. By the induction hypothesis, we
      know that both $e_1\evaluate t_1$ and $e_2\evaluate t_2$ preserve and thus
      $t_1:\Task\tau_1$ and $t_2:\Task\tau_2$. Therfore
      $t_1\And t_2:\Task(\tau_1\times\tau_2)$\\

  \noindent\textbf{Case} $\userule{E-Or}$
      Given that $e_1\Or e_2:\Task\tau$, T-Or gives us that $e_1:\Task\tau$ and
      $e_2:\Task\tau$. By the induction hypothesis, we have that both
      $e_1\evaluate t_1$ and $e_2\evaluate t_2$ preserve and thus $t_1:\Task\tau$
      and $t_2:\Task\tau$. Therefore $t_1\Or t_2:\Task\tau$.\\

  \noindent\textbf{Case} $\userule{E-Xor}$
      Evaluation does not alter the expression, therefore this case holds tivially.
\end{proof}

\subsection{???}

\begin{align*}
  \forall e:\tau & e \normalise e' &\implies& \text{ either Fail}(e') \text{ or } \exists e'',i' \text{ such that } e'\handle{i'}e''&\\
  \forall e:\tau & i\in\Firsts{(e)} &\implies& e\handle{i}e'\\
  \forall e:\tau, i & e\handle{i}e' &\implies& i\in\Firsts{(e)}
\end{align*}
