% !TEX root=../report.tex

\section{Overview}

\statefultrue


\subsection{Language}

\begin{grammar}
  Expressions
    & e    &::= & \lambda x:\tau.\ e   & – abstraction \\
    &      &\mid& e_1\ e_2             & – application \\
    &      &\mid& x                    & – variable \\
    &      &\mid& c                    & – constant \\
    &      &\mid& l                    & – location \\
  \addlinespace
    &      &\mid& e_1 \star e_2        & – operate \\
    &      &\mid& \If{e_1}{e_2}{e_3}   & – branch \\
    &      &\mid& \{e_1, e_2\}         & – pair \\
    &      &\mid& \{\}                 & – unit \\
  \addlinespace
    &      &\mid& \Ref e               & – reference \\
    &      &\mid& !e                   & – dereference \\
    &      &\mid& e_1 := e_2           & – assign \\
    &      &\mid& e_1; e_2             & – sequence \\
  \addlinespace
    &      &\mid& p                    & – pretask \\
  \addlinespace
  Constants
    & c    &::= & B                    & – boolean \\
    &      &\mid& I                    & – integer \\
    &      &\mid& S                    & – string \\
  \addlinespace
  Pretasks
    & p    &::= & \Edit e              & – valued editor \\
    &      &\mid& \Enter \beta         & – unvalued editor \\
    &      &\mid& \Store e             & – stored editor \\
  \addlinespace
    &      &\mid& \Fail                & – fail task \\
  \addlinespace
    &      &\mid& e_1 \Then e_2        & – step \\
    &      &\mid& e_1 \Next e_2        & – user step \\
  \addlinespace
    &      &\mid& e_1 \And e_2         & – composition \\
  \addlinespace
    &      &\mid& e_1 \Or e_2          & – choice \\
    &      &\mid& e_1 \Xor e_2         & – user choice \\
\end{grammar}



\newpage
\subsection{Typing}

\begin{grammar}
  Types
    & \tau &::= & \tau_1 \to \tau_2    & – function type \\
    &      &\mid& \tau_1 \times \tau_2 & – product type \\
    &      &\mid& \Unit                & – unit type \\
    &      &\mid& \Reference \tau      & – reference type \\
    &      &\mid& \Task \tau           & – task type \\
    &      &\mid& \beta                & – basic type \\
    &      &\mid& \alpha               & – universal type \\
  Basic types
    &\beta &::= & \Bool                & – boolean type \\
    &      &\mid& \Int                 & – integer type \\
    &      &\mid& \String              & – string type \\
\end{grammar}

\begin{equation*}
  \boxed{\RelationT}
\end{equation*}

\begin{mathpar}
  \userule{T-Edit} \qquad \userule{T-Fill} \qquad \userule{T-Store} \\
  \userule{T-Fail} \\
  \userule{T-Then} \qquad \userule{T-Next} \\
  \userule{T-And} \\
  \userule{T-Or} \qquad \userule{T-Xor}
\end{mathpar}



\subsection{Evaluating}

\begin{grammar}
  Values
    & v    &::= & \lambda x.\ e        & – abstraction \\
    &      &\mid& \{v_1, v_2\}         & – pair value \\
    &      &\mid& \{\}                 & – unit \\
    &      &\mid& c                    & – constant \\
    &      &\mid& l                    & – location \\
    &      &\mid& t                    & – task \\
  Tasks
    & t    &::= & \Edit v              & – valued editor \\
    &      &\mid& \Enter \beta         & – unvalued editor \\
    &      &\mid& \Store l             & – stored editor \\
    &      &\mid& \Fail                & – fail task \\
    &      &\mid& t_1 \Then e_2        & – step \\
    &      &\mid& t_1 \Next e_2        & – user step \\
    &      &\mid& t_1 \And t_2         & – composition \\
    &      &\mid& t_1 \Or t_2          & – choice \\
    &      &\mid& t_1 \Xor t_2         & – user choice \\
\end{grammar}

Notes:
\begin{itemize}
  \item $\Then$ and $\Next$ are strict in their first argument, lazy in their second.
    It doesn't matter what the continuation is.
  \item $\And$ and $\Or$ is strict in both arguments.
    \todo{Explain why?}
  \item $\Xor$ is lazy in both arguments.
    Evaluating one or both options of $e_1 \Xor e_2$ before the user makes a choice will result in untimely reference updates.
  % \item Although internal steps should be evaluated without user interaction,
  %   $t \Then e$ \emph{is} a task value, iff $\Value(t) = \nothing$.
  %   Take for example $t = \Enter \Int$.
\end{itemize}

\begin{equation*}
  \boxed{\RelationV}
\end{equation*}

\begin{mathpar}
  \userule{E-Edit} \qquad \userule{E-Fill} \qquad \userule{E-Store} \\
  \userule{E-Fail} \\
  \userule{E-Then} \qquad \userule{E-Next} \\
  \userule{E-And} \\
  \userule{E-Or} \qquad \userule{E-Xor}
\end{mathpar}
(Standard constructs as defined by \textcite{books/Pierce02TAPL})



\subsection{Observating}

\begin{equation*}
  \begin{array}{lcl}
    \multicolumn{3}{l}{\Value : \mathrm{Tasks} \rightharpoonup \mathrm{Values}} \\
    \Value(\Edit v)       &=& v \\
    \Value(\Enter \tau)   &=& \bot \\
    \Value(\Store l, s)   &=& s(l) \\
    \Value(\Fail)         &=& \bot \\
    \Value(t_1 \Then e_2) &=& \bot \\
    \Value(t_1 \Next e_2) &=& \bot \\
    \Value(t_1 \And t_2)  &=& \left\{
      \begin{array}{ll}
        \{v_1, v_2\}      & \when\ \Value(t_1) = v_1 \land \Value(t_2) = v_2 \\
        \bot              & \otherwise
      \end{array}
    \right. \\
    \Value(t_1 \Or t_2)   &=& \left\{
      \begin{array}{ll}
        v_1               & \when\ \Value(t_1) = v_1 \\
        v_2               & \when\ \Value(t_1) = \bot \lor \Value(t_2) = v_2 \\
        \bot              & \otherwise
      \end{array}
    \right. \\
    \Value(t_1 \Xor t_2)  &=& \bot
  \end{array}
\end{equation*}

\begin{equation*}
  \begin{array}{lcl}
    \multicolumn{3}{l}{\Succeeding : \mathrm{Tasks} \to \mathrm{Booleans}} \\
    \Succeeding(\Edit v)       &=& \True \\
    \Succeeding(\Enter \tau)   &=& \True \\
    \Succeeding(\Store l)      &=& \True \\
    \Succeeding(\Fail)         &=& \False \\
    \Succeeding(t_1 \Then e_2) &=& \Succeeding(t_1) \\
    \Succeeding(t_1 \Next e_2) &=& \Succeeding(t_1) \\
    \Succeeding(t_1 \And t_2)  &=& \Succeeding(t_1) \lor \Succeeding(t_2) \\
    \Succeeding(t_1 \Or t_2)   &=& \Succeeding(t_1) \lor \Succeeding(t_2) \\
    \Succeeding(t_1 \Xor t_2)  &=& \Succeeding(t_1) \lor \Succeeding(t_2)
  \end{array}
\end{equation*}

\todo{Is $\Succeeding$ preserved by $\normalise$?}



\newpage
\subsection{Normalising}

\begin{equation*}
  \boxed{\RelationN}
\end{equation*}

\paragraph{Step}
\begin{mathpar}
  \userule{N-ThenStay} \\
  \userule{N-ThenFail} \\
  \userule{N-ThenCont}
\end{mathpar}

\paragraph{Choose}
\begin{mathpar}
  \userule{N-OrLeft} \\
  \userule{N-OrRight} \\
  \userule{N-OrNone}
\end{mathpar}

\paragraph{Congruence}
\begin{mathpar}
  \userule{N-Next} \\
  \userule{N-And}
\end{mathpar}

\paragraph{Ready}
\begin{mathpar}
  \userule{N-Edit} \qquad \userule{N-Fill} \qquad \userule{N-Store} \\
  \userule{N-Fail} \qquad \userule{N-Xor}
\end{mathpar}

\todo{Do we need to normalise \refrule{N-And} further to a pair for nice equality properties or can we just use $\Value(t)$ to check if we have a pair?}



\subsection{Handling}

\begin{grammar}
  Events
    & h    & ::=& a                    & – action \\
    &      &\mid& \First h             & – pass to first \\
    &      &\mid& \Second h            & – pass to second \\
  Actions
    & a    & ::=& v                    & – change editor to value \\
    &      &\mid& \Clear               & – clear an editor \\
    &      &\mid& \Continue            & – continue with next task \\
    &      &\mid& \Pick r              & – pick route \\
  Routes
    & r    & ::=& \Left r              & – go left \\
    &      &\mid& \Right r             & – go right \\
    &      &\mid& \Here                & – stay here \\
\end{grammar}

\begin{equation*}
  \boxed{\RelationH}
\end{equation*}

\paragraph{Editing}
\begin{mathpar}
  \userule{H-Change} \qquad \userule{H-Clear} \qquad \userule{H-Fill} \\
  \userule{H-Store} \qquad \userule{H-Fail}
\end{mathpar}
\todo{Should $\Fail$ be a black hole or just go stuck?}

\paragraph{Continuing}
\begin{mathpar}
  \userule{H-PickLeft} \qquad \userule{H-PickRight} \\
  \userule{H-PickHere} \qquad \userule{H-Next}
\end{mathpar}

\paragraph{Passing}
\begin{mathpar}
  \userule{H-PassThen} \qquad \userule{H-PassNext} \\
  \userule{H-FirstAnd} \qquad \userule{H-SecondAnd} \\
  \userule{H-FirstOr}  \qquad \userule{H-SecondOr}
\end{mathpar}



\subsection{Driving}

\begin{equation*}
  \boxed{\RelationD}
\end{equation*}

\begin{mathpar}
  \userule{D-Main}
\end{mathpar}
