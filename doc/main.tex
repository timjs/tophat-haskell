\documentclass
  [a4paper
  ,justified
  ,nofonts
  % ,fleqn
  % ,nols
  % ,sfsidenotes
  ]{tufte-handout}


\input macros/imports
\input macros/setups
\input macros/general
\input macros/languages
\input macros/abbreviations

\input commands
\input rules

\setcounter{secnumdepth}{2}
% \setcounter{tocdepth}{1}


\title{Semantics of interactive processes}
\author{Tim Steenvoorden}


\begin{document}


\maketitle


In this document we investigate a language to model interactive processes.
We will use the term \emph{task} to denote these kind of processes.
It is an introduction explaining design choices and giving examples on the way.
By using five basic constructs,
each with its own characteristics,
we will show how to express interactive, workflow like systems using tasks.
The author claims this language is capable of expressing the majority of programs that can be written using the \ITASKS framework \cite{conf/pepm/PlasmeijerAKLNG11}.


\tableofcontents
\newpage


\input sections/introduction
\input sections/basics
\input sections/editor
\input sections/fail
\input sections/sequence
\input sections/pair
\input sections/choice
\input sections/share
% \input sections/overview


\bibliography{bibliography}
\bibliographystyle{apalike}


\end{document}
