% !TEX root=../pldi2019.tex

\section{Comparison}

In this section we compare TOP with Hoare's Communicating Sequential Processes (CSP).
We chose CSP as an example to stand for the numerous process algebras in existence, which, as far as this section is concerned, are reasonably similar.
Notions like communication with the environment and between subsystems, concurrency, and sequential composition can be found in both TOP and in process algebras.
This raises the question of how these systems relate.

We provide comparisons based on three aspects.
First, we compare the scope of the languages, as intended by the authors.
Second, we show how similar features like concurrency or communication work.
Third, we provide some example problems and illustrate how they can be solved in each language.

\subsection{Scope}

The central goal of CSP is to model the patters of behaviour of processes.
These patterns of behaviour manifest themselves in sequences of actions that actors can perform.

The central goal of TOP is to coordinate the collaboration between people who work together to reach a common goal.

CSP has a formal semantics that allows various kinds of correctness proofs, including equality of processes, and adherence to a specification.
This allows applications in program correctness, proofs of deadlock freedom, liveness, or verification of protocols.

TOP focuses less on formal correctness, and more on practical applicability.
It wants to be a language with intuitive semantics that facilitates communication between programmers and domain experts.
TOP programs should hide implementation details from domain experts while containing enough information to allow automatic generation of executable applications, including user interfaces.

\subsection{Features}

In this section we focus on certain features common to TOP and CSP, and study their respective realization.
When we point out differences, we do not argue that the different realizations are incompatible.
As a matter of fact the primitives can certainly be expressed in terms of each other, with more or less effort.
Instead, we point out how the systems emphasize certain points of view by choosing different basic building blocks.

\begin{itemize}
\item Communication with the environment: prefixing vs. editors
\item Communication between subsystems: concealment vs. shared data and monadic bind
\item Sequential composition: success vs. monadic bind
\item Concurrency, parallel composition
\item Synchronization
\item Nondeterminism
\end{itemize}

\subsection{Examples}

\begin{itemize}
\item Mutual exclusion
\item Semaphores
\item Cigarette smokers
\end{itemize}
