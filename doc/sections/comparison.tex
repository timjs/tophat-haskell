% !TEX root=../pldi2019.tex

\section{Comparison}

In this section we compare \tophat with Hoare's Communicating Sequential Processes (CSP).
We chose CSP as an example to stand for the numerous process algebras in existence.
Notions like communication with the environment and between subsystems, concurrency, and sequential composition can be found in TOP and in process algebras.
This raises the question how they relate.

Our goal is to highlight the differences and similarities between these two languages.
We provide comparisons based on three aspects.
First, we compare the scope of the languages, as intended by the authors.
Second, we show how similar features like concurrency and stepping work.
Third, we provide some example problems and illustrate how they can be solved in each language.

\subsection{Scope}

\subsection{Features}

\begin{itemize}
\item Prefixing, communication with the environment
\item Concealment, communication between subsystems
\item Sequential composition
\item Concurrency, parallel composition
\item Synchronization
\item Nondeterminism
\end{itemize}

\subsection{Examples}

\begin{itemize}
\item Mutual exclusion
\item Semaphores
\item Cigarette smokers
\end{itemize}
