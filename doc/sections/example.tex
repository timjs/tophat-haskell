% !TEX root=../pldi2019.tex

\section{Example}

\lstset{emph={p, ps, ss}}

In this section we develop a small example program to demonstrate the capabilities of \TOPHAT.
The example is a simple flight booking system.
It demonstrates communication with the environment, communication between parallel tasks, synchronisation, and input validation.

The application works as follows.
\begin{enumerate*}
  \item A user has to input a list of passengers for which to book tickets.
  \item At least one of these passengers has to be an adult.
  \item After a valid list of passengers has been input, the user has to pick seats.
  \item Only free seats may be picked.
  \item Every passenger must have exactly one seat.
  \item Multiple users should be able to book tickets at the same time.
\end{enumerate*}
For this example we assume that the host language has lists and two functions on them: intersect and difference.
For brevity, we omit the type annotations of variable bindings.

We start by defining some type aliases.
A passenger is a pair with name and age.
A seat is a pair with a row number and a seat letter.
A booking contains a list of passengers and a list of seats.
\begin{TASK}
  type Passenger = String * Int
  type Seat = Int * String
\end{TASK}

Choosing seats requires reading and updating shared information.
The list of free seats is stored in a reference called \TS{freeSeats}.
\begin{TASK}
  let freeSeats = ref [<<1,"A">>, <<1,"B">>, <<1,"C">>, ...] in
\end{TASK}

The flight booking task starts with entering a valid list of passengers,
denoted by \TS{enter (List Passenger)}.
A passenger is valid if the name is not empty and the age is at least 0.
A list of passengers is valid if each passenger is valid, and at least one of the passengers is an adult.
When the user has entered a valid list of passengers, the step after \TS{>>?} becomes enabled,
and the user can proceed to picking seats.
\begin{TASK}
  let valid = \p. not (fst p == "") /\ snd p >= 0 in
  let adult = \p. snd p >= 18 in
  let allValid = \ps. all valid ps /\ any adult ps in
  let bookFlight = enter (List Passenger) >>? \ps.
    if allValid ps then chooseSeats ps else fail in
\end{TASK}
A selection of seats is correct if every entered seat is free.
\begin{TASK}
  let correct = \ss. intersect ss !freeSeats == ss in
  let chooseSeats = \ps. enter (List Seat) >>? \ss.
    if correct ss /\ length ps == length ss
      then confirmBooking ps ss else fail in
\end{TASK}
The function \TS{confirmBooking} removes the picked seats from the shared list of free seats,
and displays the end result using the \TS{edit}-construct.
\begin{TASK}
  let confirmBooking = \ps. \ss.
    freeSeats := difference !freeSeats ss; edit <<ps, ss>> in
\end{TASK}
% It uses a function \TS{difference}, which removes all elements from the second list from the first list.

The main task appoints the \TS{bookFlight} task to three different users: \TS{u1}, \TS{u2}, and \TS{u3},
and run these tasks in parallel.
%%NOTE: parentheses make the line to long...
\begin{TASK}
  u1 @ bookFlight <&> u2 @ bookFlight <&> u3 @ bookFlight
\end{TASK}

\begin{figure}[h]
  \includegraphics[width=\columnwidth]{figures/flight-booking.png}
  \caption{
    Running web application of the flight booking example using a translation to \ITASKS,
    a \TOP implementation.
    It shows three users booking a flight simultaneously.
    The first user entered name and age and continued to picking seats.
    The second is entering details of two passengers.
    The ages are not filled in, and therefore the \TS{Continue} button is disabled.
    The third user finished a booking.
    Note the first user can't pick seats \smallcaps{1b} and \smallcaps{1c} any more.
    Also, the message bubble shows it is only allowed to enter an integer in the \TS{age} field.
  }
  \label{fig:flight-booking}
\end{figure}
