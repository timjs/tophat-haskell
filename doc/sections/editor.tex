% !TEX root=../main.tex

\section{Just an editor}

Till now we don't have any interaction with the end user.
The only thing we can create are pure task, containing a value.
It is kind of hard to claim we model \emph{interactive} workflows without any user interaction at all.
% At the core of interactive workflows are user interaction.
An end user should be able to enter information into the system,
clear it, reenter it et cetera.
This is modelled by an \emph{editor}.

Editors may or may not contain a current value.
When it has a value, it can be \emph{changed} or \emph{cleared}.
When the editor is empty, a value can be \emph{reentered}.


For this purpose, we extend our task language with two constructs:
a valued editor $\Edit v$ and an empty editor $\Empty \tau$.
\begin{grammar}
  Tasks & t & ::=& \ldots      & \\
        &   &\mid& \Edit v     & – Valued editor \\
        &   &\mid& \Empty \tau & – Empty editor \\
\end{grammar}
Valued editors contain a value $v$ and therefore inherit the type of $v$.
This is expressed by the typing rule for edit.
\begin{equation*}
  \TEdit
\end{equation*}
Empty editors do not have a value,
but, because we strive to a fully typed system,
we like to assign a type to them.

We have two options to solve this:
\begin{enumerate*}
  \item let the empty editor have a polymorphic type;
  \item annotate empty editors with a type and use that.
\end{enumerate*}
The first option sounds appealing, however, consider the following use case.
We start with an editor containing the value $42$: $\Edit 42$.
The user can change this value, as long as it is an integer,
for example to $37$: $\Edit 37$.
Clearing the value results in an empty editor: $\Empty$.
Now, is the user allowed to enter a value of some other type?
That is, can we now enter a string?
This would change the type of the editor!
Therefore we choose to annotate empty editors with a type.
The typing rule becomes:
\begin{equation*}
  \TEmpty
\end{equation*}


\subsection{Interface}

To interact with an editor,
we can present an input box in a \GUI.
The user can than change and clear the value continuously.
In a text oriented world,
we can present the user with the current value of the editor
and an input prompt which asks for a new value.


\subsection{Events}


\endinput

Editors let us view and change a value in the system.
However, an editor can also have no value at all.
This is the case when the user clears the value.
