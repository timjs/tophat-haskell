% !TEX root=../main.tex

\section{Editors: interaction with end users}

In the introduction we claimed to create a language to model \emph{interactive} workflows.
By interaction we mean communication with people using the developed system.
These \emph{end users} should be able to enter information into the system,
change it, clear it, reenter it, etc.
To do this, we introduce the concept of an \emph{editor}.

Editors may or may not contain a current value.
When an editor has a value, it can be \emph{changed} or it can be \emph{cleared}.
When it is empty, a value can be \emph{filled}.
This is depicted as a state diagram in \autoref{fig:editor-state} below.

\begin{figure}
  \centering
  \includegraphics[width=0.5\textwidth]{figures/editor-state-crop.pdf}
  \caption{Possible states of an editor and its transitions.}
  \label{fig:editor-state}
\end{figure}

One can consider editors as an abstraction over widgets in a \GUI or form fields on a webpage.
Take for example an entry for text.
Users can enter a string, change it, remove it, enter another string, and so on.
With editors we try to capture this \emph{constantly changing} nature of user input.
Also, because editors are \emph{typed},
they are only allowed to contain information of the right format.

\begin{marginfigure}
  \centering
  \includegraphics[width=0.7\marginwidth]{figures/text-entry}
  \includegraphics[width=0.7\marginwidth]{figures/switch-button}
  \includegraphics[width=0.7\marginwidth]{figures/spin-button}
  \includegraphics[width=0.7\marginwidth]{figures/map}
  \caption{
    Some examples of editors.
    From top to bottom:
    a text field for strings,
    a switch for booleans,
    a spin box for angles,
    and a map for locations.
  }
  \label{fig:editor-examples}
\end{marginfigure}

How exactly information is entered is not of great importance.
This could be an input field for a string,
a switch for a boolean,
a spin box for an angle,
or even a map with a pin for a location.
\Autoref{fig:editor-examples} depicts these examples.
In it's most banal form,
an editor is a line of text entered at a terminal and parsed to match a string, boolean, angle or location value.

\begin{margintext}{Aside: About stability}
A short side note to the iTasks guru.
Maybe you think: \enquote{I know this different states! This is stability!}
Yes and no.
I already mentioned in the introduction we won't use the notion of stability in our system.
Although the state diagram in \autoref{fig:editor-state} looks like the state diagram for stability,
it has absolutely nothing to do with it.

First, stability has \emph{three} states:
\begin{enumerate*}
  \item stable,
  \item unstable, and
  \item no value.
\end{enumerate*}
Editors have only \emph{two}:
\begin{enumerate*}
  \item valued, and
  \item unvalued.
\end{enumerate*}

Second,
it is possible for an editor to go from a valued to an unvalued state and back again,
but there is no way to mark an editor as stable.
It is a bit like using only the unstable part of below state diagram.
That is about the only resemblance.

\includegraphics[width=\marginwidth,page=2]{figures/editor-state-crop.pdf}

Third,
the state diagram in \autoref{fig:editor-state} is about \emph{editors only}.
A value in our system is always there.
Because values are just values,
as we are custom from lambda calculus and other functional programming languages,
stability does not propagate through our whole language.
Editors are the \emph{only} entities in our language that are allowed to contain a value or not.
This doesn't change anything to values in our underlaying expression language.

I can't stress it enough, so here it is again:
\begin{quote}
Editors live on the task level of our language,
values live on the expression level.
We do not change \emph{anything at all} on the expression level when introducing tasks.
\end{quote}
\end{margintext}

Thus, at the core,
an editor is a container holding a value
or holding nothing.
For this purpose, we extend our task language with two constructs:
a valued editor $\Edit e$ and an unvalued editor $\Fill \beta$.
\begin{grammar}
  Pretasks
    & p & ::=& \ldots      & \\
    &   &\mid& \Edit e     & – valued editor \\
    &   &\mid& \Fill \beta & – unvalued editor \\
\end{grammar}

Valued editors contain an expression $e$.
Therefore they inherit the type of $e$,
but embedded in the container type $\Task$.
For now, we only accept expressions of a basic type $\beta$ as a value of an editor.
This is expressed by the following typing rule for valued editors.
\begin{equation*}
  \userule{T-Edit}
\end{equation*}

Unvalued editors do not have a value,
and therefore do not wrap an expression.
Because we strive to a fully typed system,
we have two options to type unvalued editors.
\begin{enumerate*}
  \item let the unvalued editor have a polymorphic type;
  \item annotate unvalued editors with a type and use that. \label{itm:annotate}
\end{enumerate*}

The first option sounds appealing, however, consider the following use case:
We start with an editor containing the value two: $\Edit 2$.
The user can change this value, as long as it is an integer,
for example to five: $\Edit 5$.
Clearing the value results in an unvalued editor: $\Fill$.
Now, are we allowed to enter a value of some other type?
That is, can we now enter a string?
This would change the type of the editor!

Therefore,
we need to keep track of the type of values that can be entered into an unvalued editor
and we choose option (\ref{itm:annotate}).
The typing rule becomes:
\begin{equation*}
  \userule{T-Fill}
\end{equation*}

Some examples of editors expressible in our language:
\begin{itemize}
  \item $\Edit 2$ is a valued editor which contains the integer value $2$.
  \item $\Fill \String$ is an unvalued editor,
    waiting for users to enter some string.
  \item $\Edit ((\lambda x . x)\ 5)$ is a valued editor which,
    after normalisation, will contain the value $5$.
    (We will discuss normalisation in \autoref{sec:normalisation}.)
\end{itemize}


\subsection{Events}

To change values in an editor,
we should interact with the user with some kind of interface.
In a graphical setting,
we can present the user an input box.
The user can than change and clear values continuously.
In a text oriented world,
we can print out the current value of an editor
and prompt the user for a new value
or a command to clear the editor.

To abstract away from the user interface,
we introduce an event system.
It does not matter how these events are sent to the application.
This can be by pushing a button,
entering text in an input box,
committing some text on a command line,
sending it over a web socket,
etc.

We define a new syntactic category of events $\eta$.
For now, they only contain \emph{actions} $\alpha$.
We will introduce new events in a later section.
As we already discussed before,
there are three actions that can be handled by editors:
\begin{enumerate*}
  \item filling in an unvalued editor;
  \item changing the value; or
  \item clearing the value.
\end{enumerate*}
Hereof, the first two can be merged into one action.
\begin{grammar}
  Events
    & \eta   & ::=& \alpha & – action \\
  Actions
    & \alpha & ::=& v      & – change editor to value \\
    &        &\mid& \Clear & – clear an editor \\
\end{grammar}
The value of an editor, empty or not, can be changed to a value $v$ by just sending the value as an action.
To clear a valued editor, we send the $\Clear$ action.

Handling events is done by a new semantic relation.
This relation takes a task $t$ and an event $\eta$ which results in a new task $t'$.
\todo{Add task vs pretask somewhere here...}
We write
\begin{equation*}
  \boxed{\RelationH}
\end{equation*}
Formalising our intuition from previous paragraphs,
we need three rules to describe the transitions of editors.
\begin{equation*}
  \userule{H-Change} \qquad \userule{H-Clear} \qquad \userule{H-Fill}
\end{equation*}
Note that the conditions to the right of the rules take care of typing.
They make sure the type of the entered value and the type of the editor are the same.


\subsection{Example: an interactive session}

Now, we can make a dull but interactive example of a workflow.
How could an interactive session with the user look like when we start with a simple workflow with one editor,
containing the value two?
\todo{We should probably add a rule that doesn't change a task when the event doesn't match\ldots}
\begin{align*}
    & \Edit 2 \\
  \handle{5} & \hint{change value to five} \\
    & \Edit 5 \\
  \handle{\Clear} & \hint{clear the current value} \\
    & \Fill \Int \\
  \handle{50} & \hint{enter the value fifty} \\
    & \Edit 50 \\
  \handle{\str{Bye}} & \hint{changing the value to a string won't work\ldots} \\
    & \Edit 50 \\
  \handle{10} & \hint{change value to ten} \\
    & \Edit 10
\end{align*}
